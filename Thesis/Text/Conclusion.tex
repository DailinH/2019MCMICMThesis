\subsection{Conclusions}
We proposed a Opioid Spread Model to illustrate the spread and characteristics of the reported opioid crisis. Our model is promising for the following reasons:
\begin{enumerate}
	\item In order to adapt to different requirements, we constructed a complete set of analytical Opioid Spread Model through rigorous reasoning, including both Continuous and Discrete patterns. 
	
	\item Our model is precise and easy to understand. Simulation proved that it is easy to apply to reality.
	
	\item We quantified the relationship between $\lambda$ and socio-economic features, making it possible to estimate $\lambda$ with sociology statistics.
	
	\item We proposed a threshold for the government to distribute limited management resources to control opioid use based on former statics on opioid amount.
\end{enumerate}

\subsection{Limitations and Possible Solutions}
Even though our model successfully illustrated some features of opioid spread, it still possesses the following defects:
\begin{enumerate}
	\item Our analysis of a Source is only limited to an independent county, without considering the possibility that there may exist a region consisting of several neighboring counties with similar contributions to opioid amount. In other words, we failed to consider Source Regions consisting of several nodes.
	
	\item  In our simulation process, we neglected the possibility that a Source may become a Consumer and vice versa. The whole proceidure was based on the Origins but not other Sources.
	
	\item  When analyzing how socio-economic factors influence $\lambda$, lots of promising data were missing in the provided dataset. As a result, we were only able to extract two factors(i.e. education and loneliness). Even though we fully considered the independence among different factors, it may still exert a negative influence on the fitting precision. 
\end{enumerate}



