\subsection{Data Cleansing}
Following problems occur in the provided dataset:
\begin{itemize}
	\item Missing data in the form of "**", "-", "(X)" and so on.
	\item Tags may vary in description between different years, for example in some datasets, the tag \textit{Total Households} is used, while in others \textit{Total households} and \textit{Family Households} is used.
	\item Certain tags disappear in some years. 
	\item The counties that appeared in ACS datasets and NFLIS datasets are not always unanimous.
\end{itemize}
We address these problems by:
\begin{enumerate}[step 1.]
	\item For each ACS\_xx\_5YR\_DP02.csv file, we separate the \textit{data area} with others. The \textit{data area} includes only digits, "*", "-" and "(X)". 
	\item We remove all columns in the \textit{data area} with more than ten "(X)" values; 
	\item We remove all lines in the \textit{data area} with  "*", "-" or "(X)".
	\item Remove all factor columns which did not appear all through year 2010-2016.
\end{enumerate}

\subsection{Extract Influential Factors via Prior Knowledge and Linear Regression}
When addressing the first part of the problem, we assumed that $lambda$ is a constant parameter. However, in reality, there is no doubt that the relationship between drug usage and drug storage in a county is correlated with socio-economic factors. The ACS dataset provided more than five hundred possible factors, so obviously feature extraction is needed.

\subsubsection{Prior Knowledge}
A series of hypothesis have been developed to explain the socio-economic factors related to opioid use. According to statistics summarized by the American Addiction Center and other institutes\cite{10}\cite{11}, we have
\begin{itemize}
	\item College graduates aged 26 or older battled drug addiction at lower rates than those who did not graduate from high school or those who didn’t finish college.
	\item Those who live alone cope with drug addiction more than those who don't.
	\item Those with a disability are more likely to be treated with opiods and thus more likely to become addicted.
	\item Contradictory reports exist on the influence of gender and entity/race on obtaining drug addiction. 
	\item Poverty, unemployment and family history are known risk factors of opioid misuse.
\end{itemize}
The above information implies that education, loneliness, poverty, unemployment and family addiction history are the most influential factors. Contradicting reports regarding gender and entity factors allow us to assume that they play a minor part in opioid use.

Correspondingly, in the dataset, we choose \textit{\bfseries HC03\_VC85}(low educational background), \textit{\bfseries HC03\_VC14}(Nonfamily households - Householder living alone) and \textit{\bfseries HC01\_VC103}(DISABILITY STATUS OF THE CIVILIAN NONINSTITUTIONALIZED POPULATION - Total Civilian Noninstitutionalized Population). Let them be $Y_1$, $Y_2$ and $Y_3$ Further investigation showed that \textit{\bfseries HC01\_VC103} and all other disability-relevant factors were missing and therefore removed in the data cleansing proceidure. Therefore, we attempt to obtain $\lambda$ through linear regression by fitting
\begin{equation}
\lambda = \beta_1 Y_1 + \beta_2 Y_2
\end{equation}
\subsubsection{Linear Regression}


%-------------------------------------
\begin{comment}
\subsection{Extract Influential Socio-Economic Factors via LASSO}
In order to identify the socio-economic factors that exert tremendous influence on drug use, we make use of the Least Absolute Shrinkage and Selection Operator(LASSO) method. This method was originally defined for least squares models, but LASSO regulation is easily extended to a wide variety of statistical models including linear regression.\cite{4}

We consider the data from different years separately. Suppose that there are $N$ counties in the dataset, and each of these counties consists of $p$ \textbf{ \itshape covariates} (i.e. socio-economic factors). We map these covariates to the corresponding total drug use. In otherwords, the total drug use is denoted as the \textbf{ \itshape outcome}. Let $y_i$ be the outcome and $x_i = (x_1, x_2, \cdots, x_p)^T$ be the covariate vector for the $i_{th}$ case.

The LASSO estimate is defined by\cite{7}
\begin{equation}
\hat{\beta}_{LASSO}=
\arg\min_{\beta} \sum_{i=1}^{N}(y_i - x_i^T\beta)^2 + \lambda||\beta||_1
\end{equation}

subject to
\begin{equation}
 \sum_{j=1}^{p}|\beta_j|\leq t
\end{equation}

Here, $t$ determines the level of regularization. We fit the model using the coordinate descent algorithm. \cite{6}
\end{comment}