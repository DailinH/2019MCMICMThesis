Upon investigation, we find that the data in \textit{MCM\_NFLIS\_Data.xlsx} is already in desired form. Therefore we shall mainly focus on the data processing for  \textit{ACS\_XX\_5YR.xlsx}

There are three main types of missing data mechanisms: Missing Completely at Random(MCAR), Missing at Random(MAR) and Missing not at random(NMAR). Upon investigation, we find that 

\begin{enumerate}[\bfseries Problem 1.]
	\item Missing data in the form of "**", "-", "(X)" and so on.
	\item Tags may vary in description between different years, for example in some datasets, the tag \textit{Total Households} is used, while in others \textit{Total households} and \textit{Family Households} is used, making it difficult for automatic comparison.
	\item Certain tags disappear in some years, and reappear in others. 
	\item The counties that appeared in ACS datasets and NFLIS datasets are not always unanimous.
\end{enumerate}

First we address with the missing data problem, which mainly appear in file.

\begin{enumerate}[\bfseries Step 1.]
	\item For each ACS\_xx\_5YR\_DP02.csv file, we separate the \textit{data area} with others. The \textit{data area} includes only digits, "*", "-" and "(X)". 
	\item We remove all columns in the \textit{data area} with more than ten "(X)" values.
	\item We've just evacuated the missing data in the \textit{data area} with  "*", "-" or "(X)".
	\item Remove all factor columns which did not appear all through year 2010-2016.
\end{enumerate}
