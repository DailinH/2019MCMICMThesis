\textbf{To: }Chief Administrator, DEA/NFLIS Database \\
\textbf{From: }Team \# 1901279 \\
\textbf{Date: }\today \\
\textbf{Subject: }Strategy for Countering the Opidium Crisis \\

There is an impending crisis of opioid use owing to the alarming increase of opioid prescription and its widespread use and misuse.U.S. Health Resources and Service Administration pointed out that 116 people a day die from opioid-relate opioid overdoses. To address this crisis, our team propose an analytical Opioid Spread Model to depict and analyze the spread of opioid use and its characteristics. Data from NFLIS and U.S. Census Beareau concerning five U.S. states(Ohio, Kentucky, West Virginia, Virginia and Pennsylvania) were used to simulate and test this model.

We first proposed a Continuous Opioid Spread(CDS) Model, which takes inspiration from the Porous Media Contaminant Transport Model by Tang et al. Then we made slight modifications on the CDS model to create a Discrete Opioid Spread(DDS) Model to apply it to reality. To be brief, our DDS model describes the relationship between opioid storage, usage, import/output and opioid identification cases. 

In our model, we classify all counties into Sources and Consumers basing on their opioid sources. A Source obtains opioid by production or import from cities outside the five states, and a Consumer by importing opioid from adjacent counties. A county is either a Source or a Consumer, but not both. We also define the Origin as the location where general or certain specific opioids were first produced.

We combine our model and the data from NFLTS to identify the origin of specific opioid use in . We used heroin as an example. Our result showed that 
\begin{itemize}
	\item In Kentucky, no possible origin was found.
	\item In Virginia, Richmond City and Prince William are possible origins.
	\item In Ohio, Hamilton, Cuyahoga and Lake are possible origins.
	\item In West Virginia, Berkeley, Kanawha and Harrison are possible origins.
	\item In Pennsylvania, Philadelphia, Allegheny and Luzerne are possible origins.
\end{itemize}

We also proposed a opioid identification threshold, such that we can divide counties into three levels relative to the emergency to conduct strict opioid control over them. Useding Grey Method, we predicted which county will reach the threshold in the future. Our prediction showed that Erie in Ohio and Ashtabula in Ohio will surpass this theshold in 2021 if no immediate control is imposed upon them.

Based on our model, we analyzed the relationship between socio-economic factors and the opioid use trend. Existing hypothesis mainly attribute the opioid use to the following factors:
\begin{itemize}\setlength{\itemsep}{0pt}
	\item Education
	\item Loneliness
	\item Disability
	\item Gender/Race
	\item Poverty/Unemployment/Family history of opioid misuse
\end{itemize}

Basing on data extracted from U.S. Census Bureau's website, we used the LASSO method to roughly observe among over five hundred features how each contributes to $\lambda$, an attribute in our model characterizing the relationshiop between opioid usage and opioid storage. We reach the following conclusions:
\begin{itemize}
	\item We confirmed that lower educatoin background may lead to opioid use, and whether or not one lives alone relates to the possibility of opioid addiction.
	\item Gender and race did not appear to be of great influence in our analysis.
	\item Data concerning disabled population, poverty, unemployment and family history of opioid use were missing in the dataset, so how disability relates to opioid usage is unknown.
\end{itemize}

We also proposed some stratagies to counter the opioid crisis, including:
\begin{itemize}
	\item Reinforce the opioid investigation efforts in Source countries, hoping to decrease the amount of opioid storage.
	\item Reinforce the intensity of inspection and control of private vehicles passing through cross-country roads, hoping to limit the amount of opioid transported from Sources to Consumers.
	\item Actively organize propaganda to fight against opioid misuse.
\end{itemize}

We evaluated the first two strategies, and used our model to test the effectiveness and conducted sensitivity analysis to verify the validity of our model.

We will be glad to further discuss this model with you and follow through any decision you make. 