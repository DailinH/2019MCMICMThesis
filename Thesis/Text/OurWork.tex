To address the opioid crisis, we proposed a \textbf{partial difference equation} model which we call the Opioid Spread(OS) model. This model describes the spread and characteristics of the reported synthetic opioid and heroin incidents in time between counties.

First we present the precise mathematical description and deduction of the \textbf{Continuous Opioid Spread(COS) model}. Then we convey the continuous model into a Discrete Opioid Spread model. Then we will find out the possible locations where specific opioid drugs(we used Heroin for an example) might have started in each of the five states. We shall also analyze the threshold levels of opioid identifiction where the government should be extremely concerned and predict the opioid-use trends.

Then, we analyze how the U.S. Census socio-economic data provided associates with opioid use and trend-in use. To do this, we need to conduct feature extraction among 596 columns of socio-economic factors. \textbf{Data cleansing} will be applied before we proceed on identifing the factors that exert tremendous influence on the opioid use/trend-in-use. We extract the  5 most influential soio-economic factors to the total opioid use in counties using the \textbf{LASSO method}. We then use linear regression to combine them into one coefficient.

To solve the last part of the problem, we combine our former results to identify a possible strategy for countering the opioid crisis. We shall also test the effectiveness and analyze the sensitivity of model. To sum up, we will discuss our model's strength and weakness.





